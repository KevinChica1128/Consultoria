\documentclass[11pt]{beamer}
\usetheme{CambridgeUS}
\usepackage[utf8]{inputenc}
\usepackage[spanish]{babel}
\usepackage{amsmath}
\usepackage{amsfonts}
\usepackage{amssymb}
\usepackage{graphicx}
\usepackage{ragged2e}
\setbeamertemplate{navigation symbols}{} 
\author[Kevin García - Alejandro Vargas]{Kevin García \newline Alejandro Vargas }
\title[Consultoria]{Consultoria: Propuesta Cetec}


\newcommand\Wider[2][1em]{%
\makebox[\linewidth][c]{%
  \begin{minipage}{\dimexpr\textwidth+#1\relax}
  \raggedright#2
  \end{minipage}%
  }%
}


%\setbeamercovered{transparent} 
%\setbeamertemplate{navigation symbols}{} 
%\logo{} 
%\institute{} 
%\date{} 
%\subject{} 
\begin{document}
\justify
\begin{frame}
\titlepage
\end{frame}

%\begin{frame}
%\tableofcontents
%\end{frame}

\begin{frame}
\frametitle{Contenido}
\begin{itemize}
\item Problemas
\item Metodología
\item Recomendaciones
\end{itemize}
\end{frame}

\begin{frame}
\frametitle{Problemas}
\begin{itemize}
\item Problema general: El problema principal se centra en el manejo y validación de la información recogida y suministrada por los productores, la cuál se piensa que en algunos casos es errónea al no coincidir con la información final suministrada por Bucanero.
\item Problemas detectados: Los problemas detectados en el proceso de solucionar el problema principal fueron los siguientes:
\begin{itemize}
\item[-]Orden y codificación de los productores.
\item[-]Falta de información (datos faltantes).
\item[-]Falta de manejo y control estadístico de los datos.
\end{itemize}
\end{itemize}
\end{frame}

\begin{frame}
\frametitle{Metodología}
~\\Se presentará los pasos realizados para solucionar cada uno de los problemas detectados.
\begin{itemize}
\item[-]Se organizó una base de datos general, donde se suministro la información de cada productor por cada ciclo. Se trabajó con la cantidad inicial de pollos, y con la cantidad de pollos muertos.
\item[-]La base de datos se organizó con los productores presentes en los últimos ciclos suministrados (75,76 y 77) ya que estos son con los que actualmente se esta trabajando. Tuvimos un total de 48 productores y de 28 ciclos (desde el ciclo 50 hasta el 77) de los cuales no habían datos en 6 de ellos (55,56,57,65,69 y 73), por lo cual el numero final de ciclos con los cuales se trabajo fue de 22.
\end{itemize}
\end{frame}

\begin{frame}
\frametitle{Metodología}
\begin{itemize}
\item[-]Se organizaron las bases de datos con respecto a los nombres de los productores por orden alfabético y se codificaron del 1 al 48, lo cuál hace mucho más fácil el tratamiento, el seguimiento y la administración de los datos.
\item[-]Para cada productor se realizo un intervalo de credibilidad para la cantidad de pollos muertos con una confianza del 95\%, es decir, que el 95\% de las veces, la cantidad real de pollos muertos que tenga el productor va a estar en el intervalo generado.
\item[-]Se obtuvieron probabilidades predictivas, las cuales se interpretan como la probabilidad de que en un nuevo ciclo bajo las mismas condiciones, el productor tenga mas pollos muertos de los que tuvo.
\end{itemize}
\end{frame}

\begin{frame}
\frametitle{Recomendaciones}
\begin{itemize}
\item[-]Se recomienda codificar los productores por orden alfabético para facilitar el manejo de los datos.
\item[-]Cuando un productor no participe en un ciclo se recomienda no eliminarlo, simplemente se deja todo en blanco en dicho ciclo para ese productor, esto facilita que todo el proceso y los datos en general continúen con la misma forma y la misma longitud para evitar confusiones posteriores.
\item[-]Cuando un productor nuevo vaya a ingresar, asignarle el código siguiente, en nuestro caso si entra un nuevo productor se le asignará el código 49, y con este se tienen dos opciones, dejarlo de ultimo siempre sin importar el nombre que tenga, u organizarlo en la base de datos con el orden alfabético pero así mismo se debe editar lo demás (incluirlo en las demás bases) para seguir con la misma organización para todos; la segunda opción sería lo ideal pero demanda mas tiempo.
\end{itemize}
\end{frame}
\end{document}