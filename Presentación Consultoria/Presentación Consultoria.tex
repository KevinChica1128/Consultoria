\documentclass[11pt]{beamer}
\usetheme{CambridgeUS}
\usepackage[utf8]{inputenc}
\usepackage[spanish]{babel}
\usepackage{amsmath}
\usepackage{amsfonts}
\usepackage{amssymb}
\usepackage{graphicx}
\usepackage{ragged2e}
\setbeamertemplate{navigation symbols}{} 
\author[Kevin García - Alejandro Vargas]{Kevin García \newline Alejandro Vargas }
\title[Consultoria]{Consultoria: Propuesta Cetec}


\newcommand\Wider[2][1em]{%
\makebox[\linewidth][c]{%
  \begin{minipage}{\dimexpr\textwidth+#1\relax}
  \raggedright#2
  \end{minipage}%
  }%
}


%\setbeamercovered{transparent} 
%\setbeamertemplate{navigation symbols}{} 
%\logo{} 
%\institute{} 
%\date{} 
%\subject{} 
\begin{document}
\justify
\begin{frame}
\titlepage
\end{frame}

%\begin{frame}
%\tableofcontents
%\end{frame}

\begin{frame}
\frametitle{Contenido}
\begin{itemize}
\item Problemas
\item Tratamiento de datos
\item Metodología y análisis de datos
\item Recomendaciones
\end{itemize}
\end{frame}

\begin{frame}
\frametitle{Problemas}
\begin{itemize}
\item Problema general: El problema principal se centra en el manejo y validación de la información recogida y suministrada por los productores, la cuál se piensa que en algunos casos es errónea al no coincidir con la información final suministrada por Bucanero.
\item Problemas detectados: Los problemas detectados en el proceso de solucionar el problema principal fueron los siguientes:
\begin{itemize}
\item[-]Orden y codificación de los productores.
\item[-]Falta de información (datos faltantes).
\item[-]Falta de manejo y control estadístico de los datos.
\end{itemize}
\end{itemize}
\end{frame}

\begin{frame}
\frametitle{Tratamiento de datos}
\begin{itemize}
\item[-]Se creó una base de datos general, donde se suministró la información de cada productor por cada ciclo. Se trabajó con la cantidad inicial de pollos, y con la cantidad de pollos muertos.
\item[-]La base de datos se organizó con los productores presentes en los últimos ciclos suministrados (75,76 y 77) ya que estos son con los que actualmente se esta trabajando. Tuvimos un total de 48 productores y de 28 ciclos (desde el ciclo 50 hasta el 77) de los cuales no habían datos en 6 de ellos (55,56,57,65,69 y 73), por lo cual el numero final de ciclos con los cuales se trabajo fue de 22.
\end{itemize}
\end{frame}

\begin{frame}
\frametitle{Tratamiento de datos}
\begin{itemize}
\item[-]Se organizaron las bases de datos con respecto a los nombres de los productores por orden alfabético y se codificaron del 1 al 48, lo cuál hace mucho más fácil el tratamiento, el seguimiento y la administración de los datos.
\end{itemize}
\end{frame}

\begin{frame}
\frametitle{Tratamiento de datos}
\begin{figure}[!h]
        \includegraphics[width=12.3cm]{imagenes/BD.png}
        \label{figura1}
\end{figure}
\end{frame}


\begin{frame}
\frametitle{Metodología y análisis de datos}
\begin{itemize}
\item[-]Con la cantidad inicial de pollos y la cantidad de pollos muertos por ciclo para cada productor se automatizó la proporción de pollos muertos.
\item[-]Para cada productor se realizó un intervalo de credibilidad para la proporción de pollos muertos con una credibilidad del 95\%, es decir, que el 95\% de las veces, la proporción real de pollos muertos que tenga el productor va a estar en el intervalo generado. Esto se realizó para cada ciclo, en otras palabras, cada productor tendrá tantos intervalos como ciclos.
\end{itemize}
\end{frame}

\begin{frame}
\frametitle{Metodología y análisis de datos}
\begin{figure}[!h]
        \includegraphics[width=12.3cm]{imagenes/IC.png}
        \label{figura1}
\end{figure}
\end{frame}

\begin{frame}
\frametitle{Metodología y análisis de datos}
\begin{itemize}
\item[-]Posteriormente, con todos los intervalos de cada productor, se generaron tres intervalos finales ó generales con los cuales se puede evaluar el desempeño del productor en todos los ciclos. El primero, es el intervalo ``flexible", el cuál se construyo con el mínimo de todos los limites inferiores y el máximo de todos los limites superiores; el segundo, es el intervalo ``exigente", el cuál se construyó con el máximo de todos los limites inferiores y el mínimo de todos los limites superiores, esto lo que hace es disminuir significativamente la longitud del intervalo. El tercer intervalo, lo denominamos intervalo ``general", el cuál su construcción fue un poco distinta; se sumaron todas las cantidades iniciales y todas las cantidades de pollos muertos, obteniendo una sola proporción final (esta se pude interpretar como la proporción de pollos muertos que ha tenido el productor uniendo todos los ciclos), a esta proporción final se le realizó el intervalo de credibilidad.
\end{itemize}
\end{frame}

\begin{frame}
\frametitle{Metodología y análisis de datos}
\begin{figure}[!h]
        \includegraphics[width=12.3cm]{imagenes/IF.png}
        \label{figura1}
\end{figure}
\end{frame}

\begin{frame}
\frametitle{Metodología y análisis de datos}
\begin{itemize}
\item[-]Para la evaluación de los ciclos también se decidió hacer descriptivas, así se puede saber a ciencia cierta si un ciclo tuvo complicaciones y basado en esto no ser injustos con algunos productores.
\item[-]Se realizó un gráfico que nos indica en cada ciclo que proporción de pollos se le murieron a cada productor, por ende, los productores que tienen mas picos en su gráfica indica que la mayoría del tiempo tiene proporciones grandes de muertes
\end{itemize}
\end{frame}

\begin{frame}
\frametitle{Metodología y análisis de datos}
\begin{figure}[!h]
        \includegraphics[width=12.3cm]{imagenes/GM.png}
        \label{figura1}
\end{figure}
\end{frame}

\begin{frame}
\frametitle{Metodología y análisis de datos}
\begin{itemize}
\item[-]Finalmente, se calculó la probabilidad de que cada productor tenga una proporción de pollos muertos mayor al 5\% en más de la mitad de los ciclos, esta probabilidad sirve para evaluar al productor individualmente.
\item[-]Además, se calculó la probabilidad de que en cada ciclo más de la mitad de los productores tengan una proporción de pollos muertos mayor al 5\%, está probabilidad sirve para evaluar cuales han sido los ciclos más críticos y analizar que ocurrió en ese ciclo en particular.
\end{itemize}
\end{frame}


\begin{frame}
\frametitle{Recomendaciones}
\begin{itemize}
\item[-]Se recomienda codificar los productores por orden alfabético para facilitar el manejo de los datos.
\item[-]Cuando un productor no participe en un ciclo se recomienda no eliminarlo, simplemente se deja todo en blanco en dicho ciclo para ese productor, esto facilita que todo el proceso y los datos en general continúen con la misma forma y la misma longitud para evitar confusiones posteriores.
\item[-]Cuando un productor nuevo vaya a ingresar, asignarle el código siguiente, en nuestro caso si entra un nuevo productor se le asignará el código 49, y con este se tienen dos opciones, dejarlo de ultimo siempre sin importar el nombre que tenga, u organizarlo en la base de datos con el orden alfabético pero así mismo se debe editar lo demás (incluirlo en las demás bases) para seguir con la misma organización para todos; la segunda opción sería lo ideal pero demanda mas tiempo.
\end{itemize}
\end{frame}

\begin{frame}
\frametitle{Recomendaciones}
\begin{itemize}
\item[-]Hacer un seguimiento anual de los productores para evaluar cuales son los mas eficientes y cuales presentan problemas en la producción (gran porcentaje de pollos muertos).
\item[-]Analizar cada ciclo con el fin de detectar problemas en la producción a tiempo y solucionarlos.
\item[-]Realizar una comparación a lo largo del tiempo de los diferentes años para identificar posibles factores que puedan hacer que un año sea mas productivo que otro.
\end{itemize}
\end{frame}
\end{document}